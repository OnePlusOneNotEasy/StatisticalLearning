\documentclass[12pt, a4paper, oneside]{ctexart}
\usepackage{amsmath, amsthm, amssymb, bm, color, framed, graphicx, hyperref, mathrsfs}

\title{\textbf{课程作业}}
\author{杨佳乐}
\date{\today}
\linespread{1.5}
\definecolor{shadecolor}{RGB}{241, 241, 255}
\newcounter{problemname}
\newenvironment{problem}{\begin{shaded}\stepcounter{problemname}\par\noindent\textbf{题目\arabic{problemname}. }}{\end{shaded}\par}
\newenvironment{solution}{\par\noindent\textbf{解答. }}{\par}
\newenvironment{note}{\par\noindent\textbf{题目\arabic{problemname}的注记. }}{\par}

\begin{document}

\maketitle

\begin{problem}
3.6
\end{problem}
\begin{solution}
由贝叶斯法则可知,$\beta$的后验概率$P(\beta|D)$和似然度乘以先验概率$P(D|\beta)P(\beta)$成正比,边缘概率$P(D)$为定值。$P(D|\beta)\sim N(X^T\beta,\sigma^2I)$,$P(\beta)\sim N(0,\tau I)$。\par
$$log(P(D|\beta)P(\beta))\propto{-\frac{(y-X\beta)^T(y-X\beta)}{2\sigma^2}-\frac{\beta^T\beta}{2\tau}}$$
$$\propto{-(\frac{1}{2\sigma^2}\sum_{i=1}^{N}[y_i-\beta_0-\sum_{j=1}^{p}x_{ij}\beta_j]^2+\frac{1}{2\tau}\sum_{j=1}^{p}\beta_j^2)}$$
$$\beta_{ridge}=argmax_{\beta}{-(\frac{1}{2\sigma^2}\sum_{i=1}^{N}[y_i-\beta_0-\sum_{j=1}^{p}x_{ij}\beta_j]^2+\frac{1}{2\tau}\sum_{j=1}^{p}\beta_j^2)}$$
两边同时乘以$2\sigma^2$,令$ \lambda =\frac{\sigma^2}{\tau}$,得到两个方差相加的形式。\par
因为两个正态分布概率密度函数相乘结果还是服从正态分布,$\beta$取当右边为最大值时的值,这时候$\beta$为均值,也为众数。
\end{solution}

\begin{problem}
3.7
\end{problem}
\begin{solution}
题目3.7的解法跟3.6完全一样,只是3.6中$\beta$的方差为$\tau$,3.7中$\beta$的方差为$\tau^2$。
\end{solution}


\begin{problem}
3.8
\end{problem}
\begin{solution}
假设$Q=(q_0,q_1,...,q_p)(N\times P), X =(e,x_1,x_2,...,X_p)=QR$,
让$q_0=\frac{e}{\sqrt{N}}, r_{00}=\sqrt{N}$。\par
当$1\le{j}\le{p}$时,可以得到$\bar{q_j}=\frac{e^{T}q_j}{N}=\frac{q_0^{T}q_j}{\sqrt{N}}=0$
$x_j=\sum_{k=0}^{j}r_{kj}q_{k}$,$\bar{x_j}=\frac{r_{0j}}{\sqrt{N}}$。可以推出$x_j-\bar{x_j}e=\sum_{k=1}{p}r_{kj}q_{k}$。\par
$R_2$是R右下方一个分块矩阵$(p\times p)$,则$R = \begin{pmatrix} \sqrt{N} & \sqrt{N}(\bar{x_1},...,\bar{x_p}) \\ 0 & R_2 \end{pmatrix}$。\par
所以可以推出$Q_2R_2=\tilde{X}=UDV^T$。$Q_2$是一组基,可以看出它张成的空间和$U$一样。
\end{solution}






\end{document}