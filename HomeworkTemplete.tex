\documentclass[12pt, a4paper, oneside]{ctexart}
\usepackage{amsmath, amsthm, amssymb, bm, color, framed, graphicx, hyperref, mathrsfs}

\title{\textbf{课程作业}}
\author{杨佳乐}
\date{\today}
\linespread{1.5}
\definecolor{shadecolor}{RGB}{241, 241, 255}
\newcounter{problemname}
\newenvironment{problem}{\begin{shaded}\stepcounter{problemname}\par\noindent\textbf{题目\arabic{problemname}. }}{\end{shaded}\par}
\newenvironment{solution}{\par\noindent\textbf{解答. }}{\par}
\newenvironment{note}{\par\noindent\textbf{题目\arabic{problemname}的注记. }}{\par}

\begin{document}

\maketitle

\begin{problem}
4.1
\end{problem}
\begin{solution}
使用拉格朗日乘子法可以将其转换为标准特征值问题,问题可表示为下面形式:
$$ max_\alpha\qquad{\alpha^{T}B\alpha} $$
$$ s.t.\qquad\alpha^{T}B\alpha=1$$
转换成拉格朗日形式:
$$ L(\alpha, \lambda)=\alpha^{T}B\alpha+\lambda(\alpha^{T}W\alpha-1)$$
分别对$\alpha, \lambda$求偏导:
$$ \frac{\partial L(\alpha, \lambda)}{\partial \alpha}=2B\alpha+2\lambda W\alpha$$
$$ \frac{\partial L(\alpha, \lambda)}{\partial \lambda} = \alpha^{T}W\alpha-1$$
使L函数对$\alpha$的偏导数等于0,可以推导出:
$$ -B\alpha=\lambda W\alpha$$
$$ -W^{-1}B\alpha=\lambda\alpha$$
即为$Ax=\lambda x$的标准形式。当$\alpha$取$-W^{-1}B$的特征向量,$\lambda$取其特征值时,可以使目标函数最大。

\end{solution}

\begin{problem}
4.2
\end{problem}
\begin{solution}
(a) \par
多元正态分布的概率密度函数可以成:
$$ p_k(x)=\frac{1}{(2\pi)^{p/2}|\Sigma|^{1/2}}e^{-\frac{1}{2}(x-\mu_k)^{T}\Sigma^{-1}(x-\mu_k)}$$
对其取对数:
$$ log(p_k(x))=-\frac{1}{2}ln(|\Sigma|)-\frac{p}{2}ln(2\pi)-\frac{1}{2}(x-\mu_k)^{T}\Sigma^{-1}(x-\mu_k)$$
所以判别方程:
$$ \delta_k(x)=log(p_k(x))+ln(\pi_k)$$
以上判别式假设不同种类样本的协方差都为$\Sigma$,可以将常数项丢弃,得到简化后的判别式:
$$ \delta_k(x)=-\frac{1}{2}(x-\mu_k)^{T}\Sigma^{-1}(x-\mu_k)+ln(\pi_k)$$
将第一项展开:
$$ \delta_k(x)=-\frac{1}{2}(x^{T}\Sigma^{-1}x-x^{T}\Sigma^{-1}\mu_k-\mu_k^T\Sigma^{-1}x+\mu_k^T\Sigma^{-1}\mu_k)+ln(\pi_k)$$
$x^{T}\Sigma^{-1}x$在$k$为每一个值的判别式中都一样可以省略,因为协方差矩阵使对称的,所以$x^{T}\Sigma^{-1}\mu_k=\mu_k^T\Sigma^{-1}x$,最后得到:
$$ \delta_k(x)=x^{T}\Sigma^{-1}\mu_k-\frac{1}{2}\mu_k^T\Sigma^{-1}\mu_k+ln(\pi_k)$$
假设$x$的标签是2,则$\delta_2(x)>\delta_1(x)$,再将$\pi_1=\frac{N_1}{N},\quad\pi_2=\frac{N_2}{N}$带入,即可推出题目给的不等式。\par
(b)\par
要使平方损失最小,需要满足$ \beta^* =(X^{T}X)^{-1}X^{T}y$,所以:
\begin{equation}
    X^{T}X\beta^* = X^T y
\end{equation} 
等式左边:
$$ X^T X = \begin{bmatrix} 1 & \cdots & 1 & \cdots & 1 \\
        x_1 & \cdots & x_{N_1} &\cdots & x_{N_1+N_2}\end{bmatrix} 
        \begin{bmatrix} 1 & x_1 \\
                       \vdots & \vdots \\
                       \1 & x_{N_1} \\
                       \vdots & \vdots \\
                       \1 & x_{N_1+N_2}\end{bmatrix}$$
$$ = \begin{bmatrix}
        N & \sum_{i=1}^N x_i^T \\
        \sum_{i=1}^N x_i & \sum_{i=1}^N x_i x_i^T
\end{bmatrix}$$
其中$\sum_{i=1}^N x_i = \sum_{i=1}^{N_1} x_i + \sum_{i=N_1+1}^N x_i = N_1\mu_1 + N_2\mu_2$。同理可得$\sum_{i=1}^N x_i^T=N_1\mu_1^T + N_2\mu_2^T$。对于$(X^TX)_{2,2}$,引入协方差矩阵:
$$\Sigma = \frac{1}{N-2}(\sum_{i=1}^{N_1}(x_i-\mu_1)(x_i-\mu_1)^T +  \sum_{i=N_1+1}^{N}(x_i-\mu_2)(x_i-\mu_2)^T)$$
$$ = \frac{1}{N-2}(\sum_{i=1}^{N_1}(x_i-\mu_1)(x_i^T-\mu_1^T) +  \sum_{i=N_1+1}^{N}(x_i-\mu_2)(x_i^T-\mu_2^T))$$
$$ = \frac{1}{N-2}(\sum_{i=1}^{N_1}(x_i x_i^T-x_i\mu_1^T-\mu_1 x_i^T + \mu_1\mu_1^T) +  \sum_{i=N_1+1}^{N}(x_i x_i^T-x_i\mu_2^T-\mu_2 x_i^T + \mu_2\mu_2^T))$$
其中:
$$ \sum_{i=1}^{N_1} x_i\mu_1^T =  \sum_{i=1}^{N_1} \begin{bmatrix} x_{i1}\mu_{11} & x_{i1}\mu_{12} & \cdots & x_{i1}\mu_{1n} \\ \vdots & \ddots & & \vdots \\ x_{in}\mu_11 & \cdots & \cdots & x_{in}\mu{1n} \end{bmatrix}$$
$$ =  \begin{bmatrix} N_1 \mu_{11}^2 & N_1 \mu_{11}\mu_{12} & \cdots & N_1 \mu_{11}\mu_{1n} \\ \vdots & \ddots & & \vdots \\ N_1 \mu_{1n}\mu_11 & \cdots & \cdots & N_1 \mu_{1n}^2 \end{bmatrix} $$
$$ = N_1 \mu_1 \mu_1^T $$
同理可得:$ \sum_{i=1}^{N_1}\mu_1 x_i^T = N_1\mu_1\mu_1^T, \sum_{i=N_1+1}^{N}x_i\mu_2^T=N_2\mu_2\mu_2^T, \sum_{i=N_1+1}^{N}\mu_2 x_i^T = N_2\mu_2\mu_2^T$\par
再带入前面的方程可得:
$$\sum_{i=1}^{N} x_i x_i^T = (N-2)\Sigma+N_1\mu_1\mu_1^T+N_2\mu_2\mu_2^T$$
(1)等式的右边,由题目所给条件,$1-N_1$是误分类点,其函数值是$-\frac{N}{N_1}$,$N_1-N$(共$N_2$个点)是正确分类点,其函数值是$\frac{N}{N_2}$:
$$ X^T y = \begin{bmatrix}
               1 & \cdots & 1 & 1 & \cdots & 1 \\
               x_1 & \cdots & x_{N_1} & x_{N_1+1} & \cdots & x_N
\end{bmatrix} \begin{bmatrix}
               -\frac{N}{N_1} \\
               \vdots \\
               -\frac{N}{N_1} \\
               \frac{N}{N_2} \\
               \vdots \\
               \frac{N}{N_2}
\end{bmatrix}$$
$$ = \begin{bmatrix}
               0\\
               -\frac{N}{N_1}\sum_{i=1}^{N_1}x_i + \frac{N}{N_2}\sum_{i=N_1+1}^N x_i
\end{bmatrix}
= \begin{bmatrix}
               0 \\
               N(\mu_2 - \mu_1)
\end{bmatrix}$$
综上,(1)可以表示为:
$$
\begin{bmatrix}
     N & N_1\mu_1^T+N_2\mu_2^T \\
     N_1\mu_1+N_2\mu_2 & (N-2)\Sigma+N_1\mu_1\mu_1^T+N_2\mu_2\mu_2^T
\end{bmatrix} \begin{bmatrix}
     \beta_0 \\
     \beta
\end{bmatrix} = \begin{bmatrix}
               0 \\
               N(\mu_2 - \mu_1)
\end{bmatrix}
$$
它等价于两个方程:
\begin{equation}
    N\beta_0+(N_1\mu_1^T+N_2\mu_2^T)\beta = 0
\end{equation}
\begin{equation}
   ( N_1\mu_1+N_2\mu_2)\beta_0+((N-2)\Sigma+N_1\mu_1\mu_1^T+N_2\mu_2\mu_2^T)\beta = N(\mu_2 - \mu_1)
\end{equation}
由(2)可以推出$\beta$和$\beta_0$的关系,再带入(3)中可得:
$$
left =((N-2)\Sigma+N_1\mu_1\mu_1^T+N_2\mu_2\mu_2^T-\frac{N_1^2}{N}\mu_1\mu_1^T-\frac{2N_1N_2}{N}\mu_1\mu_2^T-\frac{N_2^2}{N}\mu_2\mu_2^T)\beta
$$
$$
= ((N-2)\Sigma + \frac{N_1}{N}(N-N_1)\mu_1\mu_1^T+\frac{N_2}{N}(N-N_2)\mu_2\mu_2^T-\frac{2N_1N_2}{N}\mu_1\mu_2^T)\beta
$$
$$
= ((N-2)\Sigma + \frac{N_1N_2}{N}(\mu_1-\mu_2)(\mu_1-\mu_2)^T)\beta
$$
由题目条件可知$\Sigma_B = \frac{N_1N_2}{N^2}(\mu_1-\mu_2)(\mu_1-\mu_2)^T$
所以:
$$
[(N-2)\Sigma+N\Sigma_B]\beta=N(\mu_2-\mu_1)
$$
(C)\par
$$
((N-2)\Sigma + \frac{N_1N_2}{N}(\mu_2-\mu_1)(\mu_2-\mu_1)^T)\beta=N(\mu_2-\mu_1)
$$
令$A=\frac{N_1N_2}{N}(\mu_2-\mu_1)^T\beta$,$A$是一个标量:
$$
(N-2)\Sigma\beta=(N-A)(\mu_2-\mu_1)
$$
所以:
$$
\beta = \frac{N-A}{N-2}\Sigma^{-1}(\mu_2-\mu_1)
$$
$\beta$的方向和$\Sigma^{-1}(\mu_2-\mu_1)$的方向一致。

\end{solution}


\begin{problem}
4.3
\end{problem}
\begin{solution}

\end{solution}

\begin{problem}
4.6
\end{problem}
\begin{solution}
(a)\par
感知机可以表示为$y_i\beta^{T}x_i^*>0$,两边同时除$\|x_i^*\|$,可以得到:
$$ y_i\beta^{T}z_i>0$$
在所有样本点$(x^*_i, y_i)$中,令$m=min\lbrace y_i\beta^{T}x_i^* \rbrace$。则$y_i\beta^{T}x_i^*\ge m$成立。两边同时除以$m$,令$\beta_{sep}=\frac{1}{m}\beta$,即可推出题目要求不等式:
$$ y_i\beta_{sep}^{T}z_i\ge 1$$
(b)\par
更新超平面公式为$\beta_{new}=\beta_{old}+y_{i}z_i$,两边同时减去$\beta{sep}$得到:
$$ \beta_{new}-\beta_{sep}=\beta_{old}-\beta{sep}+y_i z_i$$
等式两边同时平方:
$$ \|\beta_{new}-\beta_{sep}\|^2=\|\beta_{old}-\beta{sep}\|^2+y_i^2\|z_i\|^2+2y_i(\beta_{old}-\beta{sep})^{T}z_i$$
其中$y_i^2\|z_i\|^2=1$,右边第三项中,因为$(y_i,x_i)$是误分类点,所以对应的$y_i\beta_{old}^{T}z_i<0$ \par
又由(a)结论可得$y_i\beta_{sep}^{T}z_i\ge 1$,所以$2y_i(\beta_{old}-\beta{sep})^{T}z_i\le -2$。最后可以推出:
$$ \|\beta_{new}-\beta_{sep}\|^2 \le \|\beta_{old}-\beta{sep}\|^2+1-2=\|\beta_{old}-\beta{sep}\|^2-1$$
可以看出如果所有样本点可分,则最多$\|\beta_{old}-\beta{sep}\|^2$步,算法可收敛。
\end{solution}





\end{document}